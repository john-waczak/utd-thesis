\documentclass[doublespacing]{utdthesis}
% For one-and-a-half spacing, use: \documentclass[halfspacing]{utdthesis}

%%% Load any desired packages in the space below.
%%% Warning: Do not load packages that change the margins, headers, or footers!
%%%
% Optional: If you want to use Times as your font, load it here.  Note that
% although package "times" should work, it may not be the best choice.  Newer
% LaTeX distributions offer "mathptmx" and "newtxtext,newtxmath" as superior
% replacements.  You should find out which is best for your LaTeX.  (If this
% sounds confusing, you probably shouldn't try to change the font to Times.)
%\usepackage{times}
%
% Optional: If your LaTeX has microtype, use it to improve text quality:
\usepackage{microtype}
%
% Recommended: If your dissertation contains math, use the AMS packages:
\usepackage{amsmath,amssymb,amsthm}
%
% Recommended: If your dissertation needs embedded graphics, use graphicx:
\usepackage{graphicx}
%
% Recommended: If your bibliography contains web page URLs, the url package
% improves their appearance (e.g., better line breaking):
\usepackage{url}
%
% Required: To satisfy UTD's formatting requirements for citations, use the
% "natbib" package.  (Use other citation packages at your own risk; not all
% are flexible enough to meet UTD's requirements.)  If you wish to use numeric
% citations, change "authoryear" to "numbers" below.  Use the "chicago" BibTeX
% style, which most closely matches the Turabian formatting required by UTD.
% UTD mandates a blank line between each pair of bibliography entries, so set
% \bibsep as shown below.  Finally, if you are accustomed to using \cite as
% your citation macro, point it to natbib's \citep macro as shown.
\usepackage[authoryear]{natbib}
\bibliographystyle{chicago}
\setlength{\bibsep}{12pt plus 1pt minus 1pt}
\let\cite=\citep
%
% Required: If you have any wide tables or figures that need to be typeset
% in landscape, use the rotating package:
\usepackage{rotating}
%
% Optional: If you use hyperref to auto-generate hyperlinks, always load it
% LAST since it modifies everything else.  In addition, only load hyperref if
% you use pdftex or pdflatex to generate PDFs directly.  Do NOT use it if you
% use plain tex or latex to generate a DVI file.  (If you are generating DVI
% files which you then convert to PDF, you should seriously consider switching
% to pdflatex.  The DVI format loses information because it cannot support
% modern PDF document features.  Using pdflatex to generate PDFs directly
% therefore results in documents of significantly higher quality.)
\usepackage{ifpdf}
\ifpdf
  \usepackage{hyperref}
\fi
%
%%% End of packages.

%%% Define all your personal macros here (if you have any).
%
\providecommand{\hyperref}[2][]{#2}

\newenvironment{exampleclasscode}
 {\parindent=1cm\vskip0pt plus2pt minus0pt\begin{verse}}
 {\end{verse}\vskip0pt plus2pt minus0pt}
%
%%% End of personal macro definitions.


%%% The following definitions MUST come before the document begins.
%
\author{John L Waczak}
\title{Title \\ To be determined}
\thesistype{Dissertation}  % or "Thesis"
\degreefull{Doctor of Philosophy}
\degreeabbr{PhD}
\subject{Physics}
\graduationmonth{May}
\graduationyear{2023}
\prevdegrees{BS} % comma-separated list of PREVIOUS degrees

% List committee members in order.  Mark chairpersons with a "*":
\committeemember*{David Lary}
\committeemember{Person 2}
\committeemember{Person 3}
\committeemember{Person 4}
%
%%% End of definitions.


%%% Beginning of actual thesis document.

\begin{document}

\frontmatter

\signaturepage
\copyrightpage{2012} % optional

\begin{dedication} % optional
This thesis class file \\
is dedicated to my students, \\
who suffered without a proper one \\
until the present time.
\end{dedication}

\maketitle

\begin{acks}{December 2022} % date when thesis first submitted to committee
  Update required!
\end{acks}

\begin{abstract}
  Update required!  
\end{abstract}

\tableofcontents
\listoffigures % required if you have any figures
\listoftables % required if you have any tables

\mainmatter

\chapter{Autonomous sensing}
\label{c:sensing}
\section{Autonomous Hyperspectral Imaging}
\label{s:hsi}
\section{Walking Robot and Hovercraft}
\label{s:wr}

\chapter{Super Resolution}
\label{c:super resolution}
\section{Hyper Spectral Images}
\label{s:sr with hsi}
\section{Visible Images}
\label{s:sr with visible}
\section{Thermal Images}
\label{s:sr with thermal}
\section{Temporal Super Resolution: Imputation}
\label{s:sr with time}

\chapter{Atmospheric Sensing}
\label{c:atmospheric sensing}
\section{Time Series Analysis for Network Nodes}
\label{s:time series}


\chapter{Biometric Analysis}
\label{c:biometrics}
\section{Pose Analysis}
\label{s:pose}
\section{Facial Landmark Analysis}
\label{s:face}

\chapter{Audio Event Analysis}
\label{c: Audio}
\section{Gunshot Detection}
\label{s:guns}
\section{Species Identification}
\label{s:birds}

\chapter{Topological Data Analysis in Physical Measurement}
\label{c:tda}
\section{Persistent Homology}
\label{s:homology}
\section{Time Series Visibility Graphs}
\label{s:visibility}
\section{Graph Spectrum Analysis}
\label{s:spectrum}
\section{Graph Neural Networks}
\label{s:gnn}

\chapter{Scientific Machine Learning}
\label{c:sciml}
\section{Interpretable Machine Learning }
\label{s:interpretable}
\section{Sparse Identification of Nonlinear Dynamics}
\label{s:sindy}
\section{Physics Informed Neural Networks}
\label{s:pinn}
\section{SciML Applications}
\label{s:sciml applications}



\chapter{Conclusion}
\label{c:conclusion}

\appendix % required only if you have appendixes

%Begin the bibliography:

\begin{thesisbib}  % <--- THIS LINE IS REQUIRED!

  % If you use BibTeX, typically the only command between \begin{thesisbib}
  % and \end{thesisbib} is:
  %
  % \bibliography{mybibfile}
  %
  % (where "mybibfile" is the name of your .bib file).  In order to keep this
  % sample file self-contained, I've created my bibliography manually below,
  % but most people wouldn't want to do that.
\end{thesisbib}  % <-- THIS LINE IS REQUIRED!


\begin{biosketch}
Kevin W.~Hamlen began learning the basics of \LaTeX{} in the Fall of 2000 in
order to publish computer science journal articles as part of his
Ph.D.~candidacy at Cornell University.
By the completion of his degree in 2006, he had written thousands of lines of
\TeX{} code.

After completing his Ph.D., Dr.~Hamlen joined the faculty of the Computer
Science Department at The University of Texas at Dallas, and graduated his
first two Ph.D.~students (Micah Jones and Sunitha Ramanujam) in 2011.
By the graduation of his third student (Richard Wartell) in 2012, he had
concluded that a properly crafted \LaTeX{} class file for UTD theses was badly
needed to streamline future dissertation preparations.
He therefore created this one in December 2012.
\end{biosketch}


\begin{vita}  % <-- THIS LINE IS REQUIRED!
  % Replace the lines below with your CV using any formatting you wish,
  % or put nothing in this section and replace these pages with your CV
  % in the resulting PDF file.  (But you MUST include the \begin{vita}
  % and \end{vita} lines even if you intend to replace the pages, since
  % those lines are needed to put the Curriculum Vitae entry into the
  % Table of Contents.)
\end{vita}  % <-- THIS LINE IS REQUIRED!


\end{document}

